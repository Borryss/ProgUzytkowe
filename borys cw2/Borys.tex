\documentclass[12pt, letterpaper, titlepage]{article}
\usepackage[left=3.5cm, right=2.5cm, top=2.5cm, bottom=2.5cm]{geometry}
\usepackage[MeX]{polski}
\usepackage[utf8]{inputenc}
\usepackage{graphicx}
\usepackage{enumerate}
\usepackage{amsmath} %pakiet matematyczny
\usepackage{amssymb} %pakiet dodatkowych symboli
\title{Pierwszy przepis LaTeX}
\author{Borys Kutsenko}
\date{Październik 2022}
\begin{document}
\maketitle
\begin{enumerate}
\item 1,5 kg jabłek (na szarlotkę najlepiej twardych i kwaśnych, np. szara reneta)
\item 5 łyżek cukru 
\item 1/2 łyżeczki cynamonu
\item 300 g mąki
\item 250 g zimnego masła (50 g masła można zastąpić smalcem)
\item 1,5 łyżeczki proszku do pieczenia
\item 5 łyżek cukru
\item 1 łyżka cukru wanilinowego
\item 1 jajko
\item Do posypania: cukier puder
\end{enumerate}


\newpage

\section{JABŁKA}
Jabłka obrać, pokroić na ćwiartki i wyciąć gniazda nasienne. Pokroić na mniejsze kawałki i włożyć do szerokiego garnka lub na głęboką patelnię.
\section{}
Dodać cukier i cynamon i smażyć przez ok. 20 minut co chwilę mieszając, aż jabłka zmiękną i zaczną się rozpadać.
\section{CIASTO}
Do mąki dodać pokrojone w kostkę zimne masło, proszek do pieczenia, cukier i cukier wanilinowy.
\section{}
Składniki połączyć w jednolite ciasto (mikserem lub ręcznie), pod koniec dodać jajko (ciasto będzie dość miękkie).
\section{}
Podzielić je na pół i włożyć obie połówki do zamrażarki na ok. 15 minut.
\section{PIECZENIE}
Piekarnik nagrzać do 180 st C. Przygotować niedużą formę*
\section{}
Wyjąć jedną połówkę ciasta z zamrażarki, pokroić nożem na plasterki i wylepić nimi spód formy. Następnie wyłożyć na to jabłka.
\section{}
Pozostałe ciasto zetrzeć na tarce bezpośrednio na jabłka (lub pokroić ciasto na plasterki i ułożyć na wierzchu).
\section{}
Wstawić do piekarnika i piec przez ok. 50 minut lub na złoty kolor. Upieczoną szarlotkę przestudzić i posypać cukrem pudrem.
\end{document}

